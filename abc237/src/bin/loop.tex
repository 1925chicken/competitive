\documentclass[a4paper,oneside,onecolumn,11pt]{jsarticle}
\usepackage{amsmath}


\begin{document}
	ABC237-Eの問題設定で「楽しさが変化しない、または減少する$n$頂点の閉路は存在するが、上昇する$n$頂点の閉路は存在しない」ことを示す。
	\par i)$n = 2$のときは明らか。
	\par ii)$n = k,(k > 2)$まで成り立つとして、$k$頂点のときの楽しさの総和を$S$とする。
			$n = k + 1$のとき、$k$頂点の閉路から適当な辺を1本抜き、その抜いた辺の両端の頂点を$a,b$とする。常に$b$から$a$に向かうとしても一般性は失わない。
			その$a,b$の間に新たに頂点$c$を追加し、$a$と$c$,$b$と$c$をそれぞれつなげる。	これから先のケースでは全て$b \rightarrow c \rightarrow a$の順番で回るとする。\\
			$k+1$頂点で「楽しさが上昇する閉路が存在する」と仮定する。\\
			A)$H_a \leq H_c \geq H_b$の場合\\
			A.$\alpha$)$H_a > H_b$のとき、楽しさが0を超える条件は$X - 2(H_b - H_a) + 2(H_b - H_c) + (H_c - H_a) > 0$を満たすこと。\\
			左辺を整理すると$X - (H_c - H_a) > 0$となるが、$X \leq 0$かつ$H_c - H_a \geq 0$よりこれを満たすことはない。\\
			A.$\beta$)$H_a \leq H_b$のとき、楽しさが0を超える条件は$X - (H_b - H_a) + 2(H_b - H_c) + (H_c - H_a) > 0$を満たすこと。\\
			先ほどと同様にして変形すると、$X - (H_c - H_b) > 0$となり、これも条件を満たさない。\\
			B)$H_a \leq H_c \leq H_b$の場合\\
			$X - (H_b - H_a) + (H_b - H_c) + (H_c - H_a) > 0$を満たせばよいが変形すると$X > 0$となるから不可能。\\
			C)$H_a \geq H_c \leq H_b$の場合\\
			C.$\alpha$)$H_a > H_b$のとき、整理した条件式は$X - 2(H_a - H_b) - (H_a - H_c) > 0$となるが、$X \leq 0$かつ$H_a \geq H_c$かつ$H_a > H_b$より条件が満たされることはない。\\
			C.$\beta$)$H_a \leq H_b$のとき、整理した条件式は$X - (H_a - H_c) > 0$となるが成立しない。\\
			D)$H_a \leq H_c \leq H_b$のとき、整理した条件式は$X > 0$となるが、成立しない。\\
	以上から命題は示された。



\end{document}